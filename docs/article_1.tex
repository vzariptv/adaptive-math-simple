% !TeX program = tectonic
\documentclass[12pt,a4paper]{article}
    % Use unicode engine (Tectonic/XeTeX): system fonts handle Cyrillic
    \usepackage{fontspec}
    \defaultfontfeatures{Ligatures=TeX}
    \setmainfont{Times New Roman}
    \usepackage[russian]{babel}
\usepackage{amsmath}
\usepackage{amsfonts}
\usepackage{graphicx}
\usepackage{geometry}
\geometry{margin=2cm}

\title{Адаптивная система перехода между уровнями задач: экспериментальная проверка}
\author{ }
\date{}

\begin{document}

\maketitle

\section{Введение}
Современные цифровые образовательные системы требуют учёта не только правильности решений, но и дополнительных факторов, влияющих на успешность обучения. 
Ключевой задачей является разработка \textbf{многокритериальной модели оценки}, позволяющей переводить учащегося между уровнями сложности задач (Low $\rightarrow$ Medium $\rightarrow$ High) или фиксировать момент полного освоения темы.  

В данной работе представлена экспериментальная проверка модели на примере трёх студентов с различным уровнем успеваемости и мотивации.

\section{Методика: формализация показателей}

Для каждого студента в течение недели вычисляются четыре ключевых показателя.  
Все они нормируются в диапазон $[0,1]$ и учитываются при интегральной оценке.

\subsection{Точность решений (Accuracy)}
Каждая задача $j$ может быть решена с первой, второй или третьей попытки. 
Для учёта штрафов используется вектор коэффициентов:
\[
penalty\_weights = (1.0, p_2, p_3), \quad \text{где } 0 < p_3 < p_2 < 1
\]

Тогда весовой балл задачи:
\[
Score_j =
\begin{cases}
1.0, & \text{если верно с первой попытки}, \\
p_2, & \text{если верно со второй попытки}, \\
p_3, & \text{если верно с третьей попытки}, \\
0,   & \text{если не решено}.
\end{cases}
\]

Итоговая точность:
\[
Accuracy = \frac{1}{N} \sum_{j=1}^N Score_j
\]

где $N$ — количество заданных задач в течение периода.

\subsection{Скорость решения (TimeScore)}
Для каждой задачи задано эталонное время $T_{ref}^j$. 
Для оценки типичной скорости студента используется медианное время всех попыток:
\[
T_{median} = \text{median}\{T_1, T_2, \ldots, T_M\},
\]
где $T_k$ — время $k$-й попытки, $M$ — общее число попыток.

Использование медианы обеспечивает устойчивость к выбросам, вызванным техническими проблемами или внешними отвлечениями.

Нормировка:
\[
TimeScore = \min\left(1, \frac{T_{ref}^{avg}}{T_{median}}\right),
\]
где $T_{ref}^{avg}$ — среднее эталонное время по задачам данного уровня.

\subsection{Прогресс (Progress)}
Прогресс определяется как доля правильно решённых задач от всех предложенных в периоде:
\[
Progress = \frac{Solved}{Total}
\]

\subsection{Мотивация (Motivation)}
Мотивация отражает поведенческие паттерны учащегося, измеряя регулярность и вовлеченность в учебный процесс независимо от академических результатов.

Формула основана на двухкомпонентной модели:
\[
Motivation = w_{cons} \cdot Consistency + w_{eng} \cdot Engagement
\]

где веса $w_{cons}$ и $w_{eng}$ задаются администратором системы через параметр $\alpha$ (коэффициент вовлеченности):
\[
w_{cons} = \frac{1}{1 + \alpha}, \quad w_{eng} = \frac{\alpha}{1 + \alpha}
\]

\subsubsection{Компонент регулярности (Consistency)}
Измеряет постоянство учебной активности в рабочие дни:
\[
Consistency = \min\left(1, \frac{DaysActive_{work}}{5}\right)
\]
где $DaysActive_{work}$ — количество рабочих дней (понедельник–пятница) с учебной активностью.

\subsubsection{Компонент вовлеченности (Engagement)} 
Оценивает интенсивность и дополнительные усилия учащегося:
\[
Engagement = \min\left(1, \gamma_{weekend} + \gamma_{intensity} + \gamma_{distribution}\right)
\]

где компоненты определяются как:

\textbf{Выходная активность:}
\[
\gamma_{weekend} = \begin{cases}
0.4, & \text{если } DaysActive_{weekend} > 0 \\
0, & \text{иначе}
\end{cases}
\]

\textbf{Интенсивность попыток:}
\[
\gamma_{intensity} = \begin{cases}
\min\left(0.4, \frac{AttemptsCount}{DaysActive_{work} \cdot 15}\right), & \text{если } DaysActive_{work} > 0 \\
0, & \text{иначе}
\end{cases}
\]

\textbf{Распределение активности:}
\[
\gamma_{distribution} = \min\left(0.2, \frac{DaysActive_{total}}{7}\right)
\]

где:
\begin{itemize}
\item $DaysActive_{weekend}$ — количество дней активности в выходные (суббота, воскресенье)
\item $AttemptsCount$ — общее количество попыток решения задач за период
\item $DaysActive_{total}$ — общее количество дней с активностью (включая выходные)
\end{itemize}

\subsubsection{Рекомендуемые значения параметров}
Для сбалансированной оценки рекомендуется:
\[
\alpha = \frac{2}{3} \Rightarrow w_{cons} = 0.6, \; w_{eng} = 0.4
\]

Данная конфигурация обеспечивает приоритет регулярности при значимом учете дополнительных усилий.

\subsubsection{Свойства модели}
\begin{enumerate}
\item \textbf{Нормализация:} $Motivation \in [0, 1]$ для всех входных значений
\item \textbf{Независимость:} не коррелирует с академическими показателями (Accuracy, Progress)
\item \textbf{Поведенческий фокус:} измеряет только паттерны активности, не результаты
\item \textbf{Справедливость:} учитывает различные стили обучения через компонент вовлеченности
\end{enumerate}

\subsubsection{Примеры расчета}
\textbf{Высокомотивированный студент} (5 рабочих дней + выходные, 50 попыток, 7 дней активности):
\[
\begin{aligned}
Consistency &= \min(1, 5/5) = 1.0 \\
Engagement &= \min(1, 0.4 + \min(0.4, 50/(5 \cdot 15)) + \min(0.2, 7/7)) \\
&= \min(1, 0.4 + 0.4 + 0.2) = 1.0 \\
Motivation &= 0.6 \cdot 1.0 + 0.4 \cdot 1.0 = 1.0
\end{aligned}
\]

\textbf{Нерегулярный студент} (2 рабочих дня, 30 попыток, 2 дня активности):
\[
\begin{aligned}
Consistency &= \min(1, 2/5) = 0.4 \\
Engagement &= \min(1, 0 + \min(0.4, 30/(2 \cdot 15)) + \min(0.2, 2/7)) \\
&= \min(1, 0 + 0.4 + 0.057) = 0.457 \\
Motivation &= 0.6 \cdot 0.4 + 0.4 \cdot 0.457 = 0.423
\end{aligned}
\]

\subsection{Интегральная оценка}
Общий показатель:
\[
FinalScore = w_a \cdot Accuracy + w_t \cdot TimeScore + w_p \cdot Progress + w_m \cdot Motivation,
\]
где $w_a, w_t, w_p, w_m$ — веса, задаваемые администратором системы.  



\subsection{Правила перехода}
Для каждого уровня установлены два порога:
\[
\begin{cases}
FinalScore < Threshold_{min} & \Rightarrow \text{понижение уровня} \\
Threshold_{min} \leq FinalScore < Threshold_{max} & \Rightarrow \text{сохранение уровня} \\
FinalScore \geq Threshold_{max} & \Rightarrow \text{повышение уровня}
\end{cases}
\]

Если студент находится на последнем уровне (High) и превышает $Threshold_{max}$, фиксируется статус <<Тема освоена>>.

\section{Экспериментальный дизайн}
В исследовании участвовали три студента:
\begin{itemize}
    \item \textbf{Иван Зачетный} (эталонный отличник).
    \item \textbf{Пётр Залетный} (часто допускающий ошибки).
    \item \textbf{Витя Среднячок} (средний по успеваемости, но мотивированный).
\end{itemize}

Каждую неделю учащимся предлагалось 20 задач определённого уровня сложности (Low, Medium или High). 
Переход на новый уровень или возврат определялся на основе вычисленных показателей.

\section{Результаты}

\subsection{Неделя 1}
\begin{itemize}
    \item \textbf{Иван Зачетный (Low $\rightarrow$ Medium)}: 
    решил все задачи с первой попытки (Accuracy = 1.0), время стабильно близко к эталонному, активность на протяжении 4 рабочих дней. Итог: перевод на Medium.
    \item \textbf{Пётр Залетный (Low $\rightarrow$ Medium)}: 
    решил около 60\% задач, часть с третьей попытки, $Accuracy \approx 0.55$, $Progress \approx 0.6$. Активность регулярная. Переведён условно на Medium.
    \item \textbf{Витя Среднячок (Low остаётся)}: 
    точность $\approx 0.5$, $Progress \approx 0.55$, но высокая мотивация (работал все 7 дней). Не достиг минимального порога, остался на Low.
\end{itemize}

\subsection{Неделя 2}
\begin{itemize}
    \item \textbf{Иван Зачетный (Medium $\rightarrow$ High)}: 
    $Accuracy = 1.0$, $Progress = 1.0$, время улучшилось. Перевод на High.
    \item \textbf{Пётр Залетный (Medium $\rightarrow$ Low)}: 
    точность упала до $\approx 0.4$, $Progress \approx 0.45$, регулярность сохранил, но ошибок много. Возвращён на Low.
    \item \textbf{Витя Среднячок (Low $\rightarrow$ Medium)}: 
    точность $\approx 0.65$, $Progress \approx 0.7$, мотивация высокая. Переведён на Medium.
\end{itemize}

\subsection{Неделя 3}
\begin{itemize}
    \item \textbf{Иван Зачетный (High $\rightarrow$ Mastered)}: 
    $Accuracy = 1.0$, $Progress = 1.0$, время $\approx 275$ сек. Освоил тему.
    \item \textbf{Пётр Залетный (Low остаётся)}: 
    точность $\approx 0.6$, $Progress \approx 0.65$, улучшение по сравнению с предыдущей неделей, но недостаточно для перехода.
    \item \textbf{Витя Среднячок (Medium остаётся)}: 
    точность $\approx 0.7$, $Progress \approx 0.7$, высокая мотивация. Оставлен на Medium для закрепления.
\end{itemize}

\section{Детализация результатов по каждому студенту}

\subsection{Иван Зачётный (отличник)}
\textbf{Неделя 1 (уровень Low):}  
Иван решил все 20 задач с первой попытки. Среднее время $T_{avg}$ = 285 cек, что близко к эталону (300 cек).  
\[
Accuracy = 1.0, \quad TimeScore \approx 1.0, \quad Progress = 1.0, \quad Motivation = 0.8
\]
Итоговый $FinalScore$ превысил $Threshold_{max}$, студент переведён на уровень Medium.  

\textbf{Неделя 2 (уровень Medium):}  
Иван снова решил все задачи с первой попытки, среднее время даже улучшилось ($T_{avg}=275$ cек).  
Все показатели близки к 1.0, итоговый $FinalScore$ снова значительно выше $Threshold_{max}$. Иван переведён на уровень High.  

\textbf{Неделя 3 (уровень High):}  
Иван завершил тему, решив все задачи за 4 дня, сохранив максимальные показатели. Получил статус <<Тема освоена>>.  

---

\subsection{Пётр Залётный (слабый студент)}
\textbf{Неделя 1 (уровень Low):}  
Пётр решал задачи с ошибками: часть — со 2–3 попытки, несколько остались нерешёнными.  
\[
Accuracy \approx 0.55, \quad TimeScore \approx 0.9, \quad Progress \approx 0.7, \quad Motivation = 1.0
\]
Итоговый $FinalScore$ оказался около границы $Threshold_{max}$ за счёт высокой мотивации (работал почти каждый день). Поэтому система дала шанс перехода на уровень Medium.  

\textbf{Неделя 2 (уровень Medium):}  
На среднем уровне показатели ухудшились: ошибки участились, время выросло ($T_{avg} \approx 320$ cек).  
\[
Accuracy \approx 0.4, \quad TimeScore \approx 0.75, \quad Progress \approx 0.6, \quad Motivation = 0.8
\]
$FinalScore$ упал ниже $Threshold_{min}$, что потребовало понижения обратно на уровень Low.  

\textbf{Неделя 3 (уровень Low):}  
На низком уровне Пётр работал более стабильно: Accuracy вырос до $\approx 0.65$, время немного сократилось.  
Но прогресс оставался ниже нормы, поэтому студент пока остаётся на Low.  

---

\subsection{Витя Среднячок (средний студент)}
\textbf{Неделя 1 (уровень Low):}  
Витя часто решал со 2-й попытки, время нестабильное (310–340 cек).  
\[
Accuracy \approx 0.6, \quad TimeScore \approx 0.85, \quad Progress = 0.8, \quad Motivation = 1.0
\]
$FinalScore$ оказался в зоне сохранения уровня, студент остался на Low.  

\textbf{Неделя 2 (уровень Low):}  
Показатели улучшились: Accuracy $\approx 0.7$, время сократилось, Progress $\approx 0.9$.  
\[
FinalScore \geq Threshold_{max}
\]
Витя переведён на уровень Medium.  

\textbf{Неделя 3 (уровень Medium):}  
На среднем уровне результаты смешанные: Accuracy упало (около 0.55), время нестабильно (320–340 cек).  
Прогресс высокий, Motivation тоже, но $FinalScore$ оказался ниже порога перехода, поэтому студент закрепляется на Medium.  

---

\section{Выводы по индивидуальным траекториям}
- Иван показал эталонную траекторию (Low→Medium→High→Mastered).  
- Пётр оказался <<мотивационным аутсайдером>>: высокая активность не компенсировала низкую точность, система корректно вернула его на низкий уровень.  
- Витя продемонстрировал медленный, но устойчивый рост: закрепление на Low, затем успешный переход на Medium.  


\section{Обсуждение}
Результаты показывают, что модель позволяет гибко адаптировать обучение под индивидуальные траектории:
\begin{itemize}
    \item успешные учащиеся быстро проходят все уровни,
    \item студенты со слабой базой, но высокой мотивацией демонстрируют положительную динамику,
    \item при низкой точности система возвращает студента на предыдущий уровень.
\end{itemize}

Таким образом, адаптивная система учитывает не только академические результаты, но и поведенческие факторы, что повышает объективность и гибкость оценивания.

\section{Заключение}
Предложенный подход обеспечивает:
\begin{itemize}
    \item справедливую и прозрачную систему переходов,
    \item учёт когнитивных и мотивационных характеристик,
    \item возможность выявления студентов с риском <<застревания>>.
\end{itemize}

Эксперимент показал, что модель применима для школьников среднего звена и может быть расширена.

\appendix
\section{Приложение A. Пример расчёта для Петра Залётного (Неделя 2)}

\begin{itemize}
  \item Уровень: Medium (задачи \#2001--\#2020), всего $N=20$ задач.
  \item Всего попыток за неделю: $51$.
  \item Решено задач: $Solved = 16$ (по первой успешной попытке), 4 задачи не доведены до верного ответа.
  \item Среднее время попытки: $T_{avg} = 313.8\,\text{s}$.
  \item Активные дни: 5 (только будни), попыток на выходных нет.
\end{itemize}

\paragraph{Штрафы за номер успешной попытки.}
Используем вектор штрафов:
\[
penalty\_weights=(1.0,\;p_2,\;p_3),\quad p_2=0.7,\;p_3=0.4.
\]
Пусть $a_1,a_2,a_3$ — количество задач, решённых на 1/2/3-й попытке соответственно, причём
$a_1+a_2+a_3=16$ и 4 задачи не решены. Тогда взвешенная сумма баллов:
\[
\textstyle \sum_{j=1}^{N} Score_j = 1.0\cdot a_1 + 0.7\cdot a_2 + 0.4\cdot a_3 = 9.10.
\]
(это из данных недели; конкретные $a_1,a_2,a_3$ не приводим, фиксируем итоговую сумму $9.10$)

\paragraph{Accuracy.}
\[
Accuracy=\frac{1}{N}\sum_{j=1}^{N}Score_j=\frac{9.10}{20}=0.455.
\]

\paragraph{TimeScore.}
Берём усреднённый эталон по уровню: $T^{avg}_{ref}=300\,\text{s}$. Тогда
\[
TimeScore=\min\!\left(1,\;\frac{T^{avg}_{ref}}{T_{avg}}\right)
= \min\!\left(1,\;\frac{300}{313.8}\right) = 0.956.
\]

\paragraph{Progress.}
\[
Progress=\frac{Solved}{Total}=\frac{16}{20}=0.800.
\]

\paragraph{Motivation.}
Активность в 5 рабочих дней, выходных попыток нет:
\[
Motivation=\frac{DaysActive}{5}+\delta_{weekend}\cdot\beta
= \frac{5}{5} + 0\cdot0.1 = 1.0.
\]

\paragraph{Интегральная оценка и решение о переводе.}
Интегральная формула:
\[
FinalScore = w_a\cdot Accuracy + w_t\cdot TimeScore + w_p\cdot Progress + w_m\cdot Motivation.
\]

\textit{(а) Равные веса (иллюстрация чувствительности):} 
$w_a=w_t=w_p=w_m=0.25$,
\[
FinalScore^{(=)} = 0.25\cdot(0.455+0.956+0.800+1.0) = 0.803.
\]
При такой конфигурации сильная мотивация и приемлемая скорость \emph{маскируют} низкую точность.

\textit{(б) Конфигурация эксперимента (акцент на точность):} 
\[
(w_a,w_t,w_p,w_m)=(0.50,\,0.20,\,0.20,\,0.10).
\]
Тогда
\[
\begin{aligned}
FinalScore^{(\ast)} &=
0.50\cdot 0.455 + 0.20\cdot 0.956 + 0.20\cdot 0.800 + 0.10\cdot 1.0 \\
&= 0.2275 + 0.1912 + 0.1600 + 0.1000 = 0.6787.
\end{aligned}
\]
Пороговая логика уровня Medium:
\[
\begin{cases}
FinalScore < Threshold_{min}=0.70 & \Rightarrow \text{понижение уровня},\\
0.70 \le FinalScore < Threshold_{max}=0.85 & \Rightarrow \text{сохранить},\\
FinalScore \ge 0.85 & \Rightarrow \text{повысить}.
\end{cases}
\]
Так как $FinalScore^{(\ast)}=0.679<0.70$, принимается решение: \textbf{понизить уровень до Low}.
Это согласуется с наблюдаемой динамикой: высокая мотивация и нормальное время не компенсируют
систематически низкую точность (много попыток, 4 задачи не доведены до верного ответа).

\paragraph{Итог по неделе 2 (Пётр).}
\[
\begin{aligned}
\text{Accuracy} &= 0.455,\\
\text{TimeScore} &= 0.956,\\
\text{Progress} &= 0.800,\\
\text{Motivation} &= 1.0,\\
\text{FinalScore} &= 0.679 \,\Rightarrow \, \text{понижение (Medium}\rightarrow\text{Low)}.
\end{aligned}
\]

\medskip
\noindent\textit{Замечание о настройках.} Если цель — поддерживать мотивацию и «не штрафовать» студентов при высокой активности,
можно уменьшить $w_a$ и повысить $w_m$. В нашем эксперименте, напротив, точность имела высокий приоритет,
что и привело к понижению при низком $Accuracy$.



\end{document}
